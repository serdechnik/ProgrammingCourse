\documentclass[12pt,a4paper]{report}
\usepackage[utf8]{inputenc}
\usepackage[russian]{babel}
\usepackage[OT1]{fontenc}
\usepackage{amsmath}
\usepackage{amsfonts}
\usepackage{amssymb}
\usepackage{graphicx}
\usepackage{cmap} % поиск в PDF
\usepackage{mathtext} % русские буквы в формулах
%\usepackage{tikz-uml} % uml диаграммы

% TODOs
\usepackage[%
 colorinlistoftodos,
 shadow
]{todonotes}

% Генератор текста
\usepackage{blindtext}

%------------------------------------------------------------------------------

% Подсветка синтаксиса
\usepackage{color}
\usepackage{xcolor}
\usepackage{listings}

 % Цвета для кода
\definecolor{string}{HTML}{B40000} % цвет строк в коде
\definecolor{comment}{HTML}{008000} % цвет комментариев в коде
\definecolor{keyword}{HTML}{1A00FF} % цвет ключевых слов в коде
\definecolor{morecomment}{HTML}{8000FF} % цвет include и других элементов в коде
\definecolor{captiontext}{HTML}{FFFFFF} % цвет текста заголовка в коде
\definecolor{captionbk}{HTML}{999999} % цвет фона заголовка в коде
\definecolor{bk}{HTML}{FFFFFF} % цвет фона в коде
\definecolor{frame}{HTML}{999999} % цвет рамки в коде
\definecolor{brackets}{HTML}{B40000} % цвет скобок в коде

 % Настройки отображения кода
\lstset{
language=C, % Язык кода по умолчанию
morekeywords={*,...}, % если хотите добавить ключевые слова, то добавляйте
 % Цвета
keywordstyle=\color{keyword}\ttfamily\bfseries,
stringstyle=\color{string}\ttfamily,
commentstyle=\color{comment}\ttfamily\itshape,
morecomment=[l][\color{morecomment}]{\#},
 % Настройки отображения
breaklines=true, % Перенос длинных строк
basicstyle=\ttfamily\footnotesize, % Шрифт для отображения кода
backgroundcolor=\color{bk}, % Цвет фона кода
%frame=lrb,xleftmargin=\fboxsep,xrightmargin=-\fboxsep, % Рамка, подогнанная к заголовку
frame=tblr
rulecolor=\color{frame}, % Цвет рамки
tabsize=3, % Размер табуляции в пробелах
showstringspaces=false,
 % Настройка отображения номеров строк. Если не нужно, то удалите весь блок
numbers=left, % Слева отображаются номера строк
stepnumber=1, % Каждую строку нумеровать
numbersep=5pt, % Отступ от кода
numberstyle=\small\color{black}, % Стиль написания номеров строк
 % Для отображения русского языка
extendedchars=true,
literate={Ö}{{\"O}}1
 {Ä}{{\"A}}1
 {Ü}{{\"U}}1
 {ß}{{\ss}}1
 {ü}{{\"u}}1
 {ä}{{\"a}}1
 {ö}{{\"o}}1
 {~}{{\textasciitilde}}1
 {а}{{\selectfont\char224}}1
 {б}{{\selectfont\char225}}1
 {в}{{\selectfont\char226}}1
 {г}{{\selectfont\char227}}1
 {д}{{\selectfont\char228}}1
 {е}{{\selectfont\char229}}1
 {ё}{{\"e}}1
 {ж}{{\selectfont\char230}}1
 {з}{{\selectfont\char231}}1
 {и}{{\selectfont\char232}}1
 {й}{{\selectfont\char233}}1
 {к}{{\selectfont\char234}}1
 {л}{{\selectfont\char235}}1
 {м}{{\selectfont\char236}}1
 {н}{{\selectfont\char237}}1
 {о}{{\selectfont\char238}}1
 {п}{{\selectfont\char239}}1
 {р}{{\selectfont\char240}}1
 {с}{{\selectfont\char241}}1
 {т}{{\selectfont\char242}}1
 {у}{{\selectfont\char243}}1
 {ф}{{\selectfont\char244}}1
 {х}{{\selectfont\char245}}1
 {ц}{{\selectfont\char246}}1
 {ч}{{\selectfont\char247}}1
 {ш}{{\selectfont\char248}}1
 {щ}{{\selectfont\char249}}1
 {ъ}{{\selectfont\char250}}1
 {ы}{{\selectfont\char251}}1
 {ь}{{\selectfont\char252}}1
 {э}{{\selectfont\char253}}1
 {ю}{{\selectfont\char254}}1
 {я}{{\selectfont\char255}}1
 {А}{{\selectfont\char192}}1
 {Б}{{\selectfont\char193}}1
 {В}{{\selectfont\char194}}1
 {Г}{{\selectfont\char195}}1
 {Д}{{\selectfont\char196}}1
 {Е}{{\selectfont\char197}}1
 {Ё}{{\"E}}1
 {Ж}{{\selectfont\char198}}1
 {З}{{\selectfont\char199}}1
 {И}{{\selectfont\char200}}1
 {Й}{{\selectfont\char201}}1
 {К}{{\selectfont\char202}}1
 {Л}{{\selectfont\char203}}1
 {М}{{\selectfont\char204}}1
 {Н}{{\selectfont\char205}}1
 {О}{{\selectfont\char206}}1
 {П}{{\selectfont\char207}}1
 {Р}{{\selectfont\char208}}1
 {С}{{\selectfont\char209}}1
 {Т}{{\selectfont\char210}}1
 {У}{{\selectfont\char211}}1
 {Ф}{{\selectfont\char212}}1
 {Х}{{\selectfont\char213}}1
 {Ц}{{\selectfont\char214}}1
 {Ч}{{\selectfont\char215}}1
 {Ш}{{\selectfont\char216}}1
 {Щ}{{\selectfont\char217}}1
 {Ъ}{{\selectfont\char218}}1
 {Ы}{{\selectfont\char219}}1
 {Ь}{{\selectfont\char220}}1
 {Э}{{\selectfont\char221}}1
 {Ю}{{\selectfont\char222}}1
 {Я}{{\selectfont\char223}}1
 {і}{{\selectfont\char105}}1
 {ї}{{\selectfont\char168}}1
 {є}{{\selectfont\char185}}1
 {ґ}{{\selectfont\char160}}1
 {І}{{\selectfont\char73}}1
 {Ї}{{\selectfont\char136}}1
 {Є}{{\selectfont\char153}}1
 {Ґ}{{\selectfont\char128}}1
 {\{}{{{\color{brackets}\{}}}1 % Цвет скобок {
 {\}}{{{\color{brackets}\}}}}1 % Цвет скобок }
}


%------------------------------------------------------------------------------

\author{И.~Д.~Липанов}
\title{Программирование}
\begin{document}
\maketitle
\tableofcontents{}
\chapter{Основные конструкции языка}
%############################################################
\section{Задание 2}
\subsection{Задание}
Пользователь задает сумму денег в рублях, меньшую 100 (например, 16). Определить, как выдать эту сумму монетами по 5, 2 и 1 рубль, израсходовав наименьшее количество монет (например, 3 х 5р + 0 х 2р + 1 х 1р).

\subsection{Теоретические сведения}

%Конструкции языка, библиотечные функции, инструменты использованные при разработке приложения.
Было использовано:
\begin{enumerate}
\item[•] функции для ввода и вывода информации, образцы которых находятся в <stdio.h>
\item[•] функция считывания символа из консоли без отображения, прототип которой находятся в <conio.h>
\item[•] использовался цикл "for" для создания циклов, которые должны выполняться заданное число раз
\end{enumerate}

%Сведения о предметной области, которые позволили реализовать алгоритм решения задачи.
Для решения задачи требовалось знать некоторые стандартные функции языка С.

Необходимое наименьшее кол-во монет было найдено с помощью функций ввода-вывода. Также несоизмеримо помог тип данных "int" с помощью которого мы объявляли номиналы монет и их кол-во.
\begin{equation}
const int n=3;
const int c[n]={5,2,1};
\end{equation}


\subsection{Проектирование}
%Какие функции было решено выделить, какие у этих функций контракты, как организовано взаимодействие с %пользователем (чтение/запись из консоли, из файла, из параметров командной строки), форматы файлов и др.

Использовалась только одна функция для взаимодействия с пользователем (так как под каждую задачу я выделял отдельный проект)
\begin{enumerate}
\item[•]  Реализация задачи была решена функцией main.cpp.
\end{enumerate}

\subsection{Описание тестового стенда и методики тестирования}

%Среда, компилятор, операционная система, др.
Использовался Qt Creator 3.5.0 (opensource) с GCC 4.9.1 компилятором
Операционная система: Windows 10


%Ручное тестирование, автоматическое, статический анализ кода, динамический.
Ручное тестирование почти отсутстовало.
Для статического анализа был использован cppcheckgui версии 1.7.1 . Ошибок и предупреждений не было.
Для автоматического тестирования был использован framework qt test, с помощью которого были реализованы модульные тесты.


\subsection{Выводы}

При написание данной программы никаких трудностей не было. Никаких сложных арифметических вычислений для написания программы не требовалось.

\subsection*{Листинги}
\lstinputlisting
{../Mrx_project1/main.cpp}

%############################################################
\chapter{Циклы}
\section{Задание 1}
\subsection{Задание}

Текст содержит следующие знаки корректуры: \verb-$- - сделать красную строку, \verb-#- - удалить следующее слово, \verb-@- - удалить следующее предложение. Произвести указанную корректировку.

\subsection{Теоретические сведения}

%Конструкции языка, библиотечные функции, инструменты использованные при разработке приложения.
Было использовано:
\begin{enumerate}
\item[•] функции для ввода и вывода из файла
\item[•] конструкция "if"
\item[•] конструкция "while"
\end{enumerate}

%Сведения о предметной области, которые позволили реализовать алгоритм решения задачи.
Для решения данной задачи необходимо было уметь считывать и записывать информацию в файл, а также уметь корректировать текст.

Требовалось считывать текст из файла, делать проверку на наличие определенных символов и записывать исправленный текст в файл.

\subsection{Проектирование}
%Какие функции было решено выделить, какие у этих функций контракты, как организовано взаимодействие с %пользователем (чтение/запись из консоли, из файла, из параметров командной строки), форматы файлов и др.

Было решено выделить одну функцию:
\begin{enumerate}
\item[•] \verb-strings.c- для нахождения символов и проведения корректировки текста.
\end{enumerate}


\subsection{Описание тестового стенда и методики тестирования}

%Среда, компилятор, операционная система, др.
Использовался QtCreator с GCC компилятором
Операционная система: Windows 10


%Ручное тестирование, автоматическое, статический анализ кода, динамический.
Использовалось ручное тестирование, автоматическое тестирование не проводилось.
Ошибок и предупреждений не возникало.

\subsection{Тестовый план и результаты тестирования}

Все тесты были пройдены успешно: полученный результат, совпал с ожидаемым.


\subsection{Выводы}

В ходе написания программы не возникло никаких трудностей

\subsection*{Листинги}

\lstinputlisting
{../Mrx_project4/main4.cpp}


\chapter{Массивы}

\section{Задание 2}

\subsection{Задание}

На шахматной доске стоят три ферзя (ферзь бьет по вертикали, горизонтали и диагоналям). Найти те пары из них, которые угрожают друг другу. Координаты ферзей вводить целыми числами.

\subsection{Теоретические сведения}

%Конструкции языка, библиотечные функции, инструменты использованные при разработке приложения.
Было использовано:
\begin{enumerate}
\item[•] несколько функций для ввода и вывода информации, образцы которых находятся в <stdio.h>
\item[•] оператор "if" обеспечивающий выборочное выполнение отдельных участков кода
\end{enumerate}

%Сведения о предметной области, которые позволили реализовать алгоритм решения задачи.
Потребовались знания синтаксиса языка С.

\subsection{Проектирование}
%Какие функции было решено выделить, какие у этих функций контракты, как организовано взаимодействие с %пользователем (чтение/запись из консоли, из файла, из параметров командной строки), форматы файлов и др.

3 раза была использована адресная арифметика:
\begin{enumerate}
\item[•] \verb-((x1==x2)||(y1==y2)||(abs(x1-x2)==abs(y1-y2)))
\item[•] \verb-((x1==x3)||(y1==y3)||(abs(x1-x3)==abs(y1-y3)))
\item[•] \verb-((x2==x3)||(y2==y3)||(abs(x2-x3)==abs(y2-y3)))
\end{enumerate}

\subsection{Описание тестового стенда и методики тестирования}
%Среда, компилятор, операционная система, др.
Использовался QtCreator с GCC компилятором
Операционная система: Windows 10


%Ручное тестирование, автоматическое, статический анализ кода, динамический.
Ручное тестирование присутствовало.
Для статического анализа был использован cppcheckgui. Ошибок и предупреждений не было.
Для автоматического тестирования был использован framework qt test, с помощью которого были реализованы модульные тесты.


\subsection{Выводы}

Проблемы были в адресной арифметике, но после некоторых изменений в местоположениях указателей результат был схож с предпологаемым.

\subsection*{Листинги}

\lstinputlisting
{../Mrx_project2/main.cpp}

\chapter{Арифметика}

\section{Задание 1}

\subsection{Задание}

Найти корни квадратного уравнения: y=ax2+bx+c.

\subsection{Теоретические сведения}

%Конструкции языка, библиотечные функции, инструменты использованные при разработке приложения.
Было использовано:
\begin{enumerate}
\item[•] функции для ввода-вывода информации, образцы которых находятся в <stdio.h>
\item[•] функция вычисления квадратного корня, образец которой находятся в <math.h>
\end{enumerate}

%Сведения о предметной области, которые позволили реализовать алгоритм решения задачи.
Для решения поставленной задачи требовалось знание основ синтаксиса языка С и умение решать квадратные уравнения.



\subsection{Проектирование}
%Какие функции было решено выделить, какие у этих функций контракты, как организовано взаимодействие с %пользователем (чтение/запись из консоли, из файла, из параметров командной строки), форматы файлов и др.

Была использована только одна функция main.cpp так как программа была выполнена в отдельном проекте.


\begin{enumerate}
\item[•] \verb-remove_characters_from_the_string- в ней реализовано удаление ненужных символов, и замена символов больших регистров на маленькие.
\item[•] \verb-find_the_longest_substring- в этой функции реализован алгоритм поиска самой длинной подстроки
\end{enumerate}

\subsection{Описание тестового стенда и методики тестирования}
%Среда, компилятор, операционная система, др.
Использовался QtCreator с GCC компилятором
Операционная система: Windows 10


%Ручное тестирование, автоматическое, статический анализ кода, динамический.
Ручное тестирование присутствовало.
Для статического анализа был использован cppcheckgui. Ошибок и предупреждений не было.
Для автоматического тестирования был использован framework qt test, с помощью которого были реализованы модульные тесты.



\subsection{Выводы}

При написание программы не возникло особых проблем.

\subsection*{Листинги}

\lstinputlisting
{../Mrx_project3/main.cpp}


\chapter{Задание на строки}

\section{Табличная функция}

\subsection{Задание}

Текст содержит многократно вложенные круглые скобки. Исправить его, оставив скобки первого уровня круглыми, второго – заменить на квадратные, третьего и последующих – на фигурные.

\subsection{Теоретические сведения}

%Конструкции языка, библиотечные функции, инструменты использованные при разработке приложения.
Было использовано:
\begin{enumerate}
\item[•] <iostream> и пространство имён std, для взаимодействия с пользователем через консоль.
\item[•] <cstring> класс с методами и переменными для организации работы со строками
\end{enumerate}

%Сведения о предметной области, которые позволили реализовать алгоритм решения задачи.

Умение считывать и записывать информацию в файл. 

\subsection{Проектирование}
%Какие функции было решено выделить, какие у этих функций контракты, как организовано взаимодействие с %пользователем (чтение/запись из консоли, из файла, из параметров командной строки), форматы файлов и др.

Была выделена одна функция:
\begin{enumerate}
\item[•] \verb-сstrings.c- для нахождения символов и проведения корректировки скобок.
\end{enumerate}

\subsection{Описание тестового стенда и методики тестирования}
%Среда, компилятор, операционная система, др.
Использовался QtCreator с GCC компилятором
Операционная система: Windows 10


%Ручное тестирование, автоматическое, статический анализ кода, динамический.
Ручное тестирование почти отсутствовало.
Для автоматического тестирования был использован framework qt test, с помощью которого были реализованы модульные тесты.

\subsection{Тестовый план и результаты тестирования}

Была произведена корректировка скобок. Все тесты были пройдены успешно: полученный результат, совпал с ожидаемым.



\subsection{Выводы}

В ходе написания программы не возникло никаких трудностей.

\subsection*{Листинги}
\lstinputlisting
{../Mrx_project5/main.cpp}


\end{document}
